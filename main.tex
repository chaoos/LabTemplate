\documentclass[a4paper]{article}

\usepackage[english]{babel}
\usepackage[utf8]{inputenc}
\usepackage{mathtools}
\usepackage{gensymb}
\usepackage{amssymb}
\usepackage{amsmath}
\usepackage{booktabs,siunitx}
\usepackage{graphicx}
\usepackage[colorinlistoftodos]{todonotes}
\usepackage{geometry}
\usepackage{float}
\usepackage[bottom]{footmisc}
\geometry{
 a4paper,
 total={170mm,257mm},
 left=30mm,
 right=30mm,
 top=20mm,
 }
\author{[[Name]]}
\title{Report - [[Experiment]]}

\begin{document}

\maketitle
\abstract 
Abstact
\section{Introduction}
Example Equation:
\begin{align}
[[EQ1]]
\label{eq:Example Equation}
\end{align}
\section{Setup}
Setup

\begin{figure}[H]
\centering
\includegraphics[width=1.0\textwidth]{plots/dT_vs_I.eps}
\caption{Plot of [[TODO]] against the time. Source: Authors' own.}
\label{fig:text}
\end{figure}

\section{Measurement Method}

Since the Peltier coefficient is proportional to the electric current density $\vec{J}$, we cannot measure it directly. It must be calculated out of measurable quantities.

\section{Measurements}
The measured value of variable1 is $[[variable1]]$.
\newline
The measured value of variable2 is $[[variable2]]$.
\newline
The measured value of variable3 is $[[variable3]]$.
\newline
The measured value of variable4 is $[[variable4]]$.
\newline
The measured value of variable5 is $[[variable5]]$.
\newline
The measured value of variable6 is $[[variable6]]$.
\newline
The measured value of variable7 is $[[variable7]]$.
\newline
The measured value of variable8 is $[[variable8]]$.
\newline
The measured value of variable9 is $[[variable9]]$.
\newline
The measured value of variable10 is $[[variable10]]$.
\newline
The polyfit of bla bla grade 1 is $[[polyfit1]]$.
\newline
The polyfit of bla bla grade 2 is $[[polyfit2]]$.
\newline
The polyfit of bla bla grade 4 is $[[polyfit3]]$.
\newline
The polyfit of bla bla is $[[polyfit4]]$.
\newline

\subsection{Errors}

The error of $I_T$ was chosen to be the change in the measured value of $I_T$ of one section in the scale of the galvanometer. When the system was in equilibrium and the potentiometer was adjusted to the zero, the value of $I_T$ was measured to be $a$. Then we changed the potentiometer until the pointer of the galvanometer was one line to the right. After that, $I_T$ was measured to be $b$. The error of $I_T$ was then calculated to be $\Delta I_T = |a-b|$. That final value of $I_T$ is $\Delta I_T = [[deltaI_T]] A$.
\newline
For the measurement of $V_{Shunt}$ and $V_p$ the same

\section{Discussion}
Discussion

\section{Appendix}

\begin{table}[H]
\centering
\begin{tabular}{r|rrr}
\hline
[[table30]]
\end{tabular}
\caption{Measured data for $T=30^\circ\text{C}$.}
\end{table}

\begin{table}[H]
\centering
\begin{tabular}{r|rrr}
\hline
[[table50]]
\end{tabular}
\caption{Measured data for $T=50^\circ\text{C}$.}
\end{table}

\begin{table}[H]
\centering
\begin{tabular}{r|rrr}
\hline
[[table80]]
\end{tabular}
\caption{Measured data for $T=80^\circ\text{C}$.}
\end{table}

\begin{table}[H]
\centering
\begin{tabular}{r|rrr}
\hline
[[table110]]
\end{tabular}
\caption{Measured data for $T=110^\circ\text{C}$.}
\end{table}

\end{document}

